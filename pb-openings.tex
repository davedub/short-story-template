\documentclass[12pt, titlepage]{article}
\usepackage[]{inputenc}
\usepackage{changepage}
% to add more info in author block
\usepackage{authblk}
\usepackage[]{inputenc}
\usepackage[OT1]{fontenc}
\usepackage{lmodern}

%to control top page only but there are other uses
% \usepackage{geometry}
% \usepackage[top=-10cm, bottom=10cm]{geometry}
\usepackage{geometry}

\hbadness=99999 % or any number >=10000
% \hbadness variable doesn't affect typography 
% it just tells TeX the threshold for printing
\hfuzz=12pt % suppress overfull hbox messages
\vfuzz=30pt % suppress overfull vbox messages

\usepackage{lipsum}

\usepackage{newunicodechar}
\newunicodechar{ʼ}{'}
% to prevent orphaned lines ...
\widowpenalty=10000
\clubpenalty=10000

% https://tex.stackexchange.com/a/552852
\DeclareFontSeriesDefault[sf]{bf}{bx}

\usepackage{sansmathfonts}
\renewcommand{\familydefault}{\sfdefault}

\usepackage{textcomp}
\usepackage{setspace}
\usepackage{hyphenat}
% \usepackage[]{inputenc}
% \usepackage{csquotes}

% Set \parskip to put 6pt between paragraphs
\setlength{\parskip}{6pt}

\usepackage{relsize}
\usepackage{fancyhdr, graphicx,lastpage} 
\pagestyle{fancy}
\fancyhf{} % clears all existing header/footer entries
\renewcommand{\headrulewidth}{0pt} % Remove line by setting width to 0
\setlength{\headheight}{52pt}% ...at least 51.60004pt
\fancyhead[C]{\ifnum\value{page}=1 \flushright\footnotesize{\textsc{{Page 1}}} \fi } 
\fancyhead[L]{\ifnum\value{page}>0 \footnotesize{\textsc{\flushleft{Prologue}}} \fi}
\fancyhead[R]{\ifnum\value{page}>1 \footnotesize{\textsc{\flushright{Page \thepage}}} \fi}


% \fancyhead[C]{\footnotesize{\textsc{Essay Template}}}
\fancyfoot[L] {\ifnum\value{page}>1 \footnotesize{\textsc{$\copyright$ \hspace{.02cm} Phil Borges }} \fi}
% \fancyfoot[C]{\footnotesize{--~\thepage~--}}
\fancyfoot[R]{\ifnum\value{page}>1 \footnotesize{\textsc{ All rights reserved}}  \fi}

\title{\Large{PB Memoir --- Untitled}\\
\vspace*{.75em}
%\large{Subtitle (if any)}\\
\vspace*{.75em}}
%\small\textit{Now available for non-commercial use sharing.}}
\author{\large{by Phil Borges\\}
\vspace{3em}
Draft as of: June 2024

\begin{center}
 \begin{minipage}{0.75\textwidth}
  \vspace{4cm}
  \begin{spacing}{1.1}
    \raggedright %
    % \setlength{\parindent}{.6cm} % indent first line
    \small{\textit{``The most beautiful thing we can experience is the mysterious. It is the source of all true art and science. He to whom the emotion is a stranger, who can no longer pause to wonder and stand wrapped in awe, is as good as dead —his eyes are closed.''}}\\
    \footnotesize\hspace*{\fill}{~~ Albert Einstein}
  \end{spacing}
\end{minipage}
\end{center}

}
\date{}

\thispagestyle{fancy}{%

\begin{document} 
\thispagestyle{fancy}{%
\maketitle

\thispagestyle{fancy}  % Optional: Clear headers/footers for subsequent blank page (if needed)
% \clearpage            % Force a new page after the title page

% \renewcommand{\headrulewidth}{1pt} % Remove line by setting width to 0
\vspace{-3.5\baselineskip} 

\setlength\parindent{20pt}
\renewcommand{\arraystretch}{1.3}
\vspace{1\baselineskip} 
\begin{flushright}
\end{flushright}

% \onehalfspacing
\begin{spacing}{1.15}
    \vspace*{1em}
\begin{center}
    \Large{Prologue}\\
% \normalsize{Approximately [X,000] words.}
\end{center}

\vspace*{1em}
Pakistan, late-August, 2001 --- three weeks before 9/11.  I am traveling with two companions on a rough gravel road in the middle a hot afternoon. My spot is in the back seat of an open air Jeep that is more than a little banged up, having taken many prior journeys like this. I watch Karim, its driver and owner, as he keeps both hands firmly on the wheel and looks through the grimy windshield, half covered in caked dust. My 15-year-old son Dax is seated beside him as we head north into the remote Hindu Kush mountains of the ``lawless'' Northern Frontier.  We have not seen another vehicle or anything that looks like a permanent dwelling for the past seventy miles. Above the crunching sounds of the Jeep's tires hitting and scattering the gravel below us, I hear a dull thud and my attention quickly shifts to Dax as he slumps forward and hits his head on the dashboard, then leans perilously close to falling off the open side. 

I frantically jump up from behind him, reach over the roll bar, and steady him.  I knew Dax was not feeling well but now, as I look down on his frail frame, I realize he must be seriously ill. In just a few hours, the intense heat and dust swirling up and into our Jeep has pushed him over an edge. Severely dehydrated and weakened by an intestinal bug he picked up a few days before, his ashen face and near lifeless form make me realize: Dax is in serious danger due to dehydration. 

Only a few days before, while we were staying with the people of an isolated tribal community in the north, getting to know them and taking their photos, everything seemed fine. But the day after we left the Kalash Valley and headed north through the rugged Hindu Kush, Dax let me know he was having bad stomach cramps. His boyish face reflected what must have been a never before felt intense pain. Over the last few days, I had hoped he would gradually feel better, with plenty of fresh air and our hosts' home-cooked food, plus a broad-spectrum `antibiotic bomb' of Ciprofloxacin. It had worked to stop his diarrhea which had prevented him from keeping anything down, or in. But now, ghostly pale and no longer able to sit up straight, he was taking deep, labored breaths.  Vibrating and rattling on this washboard road, the prospect that his strength might be slipping, making him unconscious and putting him at life-threatening risk, terrified me. Only an hour before, Karim had let me know that we were at least two days away from any reliable medical help. 

How could this be happening?  I struggled to stop the unthinkable possibility from entering my mind.  How could I be so dumb as to let Dax lose his health and possibly his life on this crazy trip.
 
By now, we had already spent nearly two full weeks with the Kalash, an animistic tribal group of 3,500 living near the Afghan border.  Dax had brought a soccer ball with him on our trip, which immediately made him a hero to the boys in the village. They had been playing with a woven ball of dried plant leaves and twigs, and seemed to be at it every day from sun up until the ball was impossible to see in the darkness and too battered to last much beyond it. Whenever Dax was not working by my side, he was with them, trading soccer tricks and moves. He had learned the charismatic magic that a soccer ball could bestow while on vacation with my wife Julee and I in Indonesia. At night, the teens there would take him to where they would gather on a nearby hilltop to sing out and dance to their traditional music. Dax was thrilled with all his new young friends and soaking up his new found celebrity. 

The Kalash have managed to keep their shamanic traditions alive despite being physically surrounded for centuries by the culture of Islam.  We had spent our last days in the Kalash territory hiking up to an 11,000-foot mountain camp to interview and photograph Janduli Kahn, one of their most respected Dehars (Shamans).  We were almost at the summer pasture where Janduli and his sons tend their goats and sheep when we realized we had accidentally left our water filter back in the Kalash village.  Evidently the clear glacial run off wasn't as pure as Yasir, our young Kalash guide, had promised. 

It was just three months ago that Julee and I decided it was time to give Dax an experience to mark his passage into adulthood.  Problem was, other than getting a driver's license, we had no traditional `rite of passage' of our own.  We had taken him on vacations to Hawaii, Mexico, and Indonesia but never far off the beaten path. We eventually decided one of my trips to a remote tribal culture would introduce him to the diverse reality of our world he had yet to discover. Dax was thrilled to go, however, our current reality wasn't the one we bargained for!  

 Nepal was actually our first choice, but there was a Maoist insurgency happening there at the time.  The decision to go to the so-called `lawless' Northwest Frontier Province of Pakistan wasn't as crazy as it seems—at least not until now. The province earned the `lawless' title not because it was considered dangerous, but because it wasn't governed by the Pakistani government. I knew the Taliban were in the area, but all I knew about them was they didn't allow girls to go to school.  How would I have known more? The horrific events of 9/11 that focused the world's attention on this area had yet to happen. 

Now, as I hold my panic in check by attempting to beat back catastrophic thoughts, our Jeep rounds a corner.  A strange log and rock structure appears in the distance. It looks like it might be a storage shed.  As we get closer it seems to be abandoned, but, in desperation, I ask Karim to pull over.  As the dust settles, I help Dax out of the Jeep and down an embankment to lie on the cool grass in the shade of a grove of trees, then run up to the peculiar structure and bang on what looks like a door.  I hear a rustling inside. There's someone in there!  Maybe they have some herbs or a local remedy—anything.  I hold my breath. The door creaks open and a smiling middle-aged man in a clean white Salwar Kameez (traditional shirt/pant combination) appears. Framed by the doorway and dark background, he looks luminous.  Speaking with a light accent he asks, ``May I help you?'' I don't exactly know what I was expecting, but it definitely wasn't him.  He's too clean and fluent in English to be in this place.   I'm stunned by the incongruity. Finally, I blurt out, ``My son is very ill. Very dehydrated.  Do you have anything that can help?''

``Where is your son?''

I point, ``Down there under the trees.''

``Yes, I'm a doctor.''

 Could this be real?  Dr. Rasheed turns out to be from Islamabad and just happens to be visiting his mother for a few days….and just happens to have an IV hydration kit.  I can't believe it. He's here in this shack in the middle of nowhere.  I feel like I've slipped into another dimension. 

There isn't room in the `house', so we carry out the IV equipment, a small bed, and set them up in the grove of trees next to Dax.  Dr. Rasheed inserts the needle in Dax's arm and starts the IV drip.  I notice a three-inch air bubble in the IV tubing.  Once again my anxiety kicks into overdrive. I frantically think, Could this cause a fatal air embolism? I don't know a thing about this guy's background.  Anxiously, I insist he stop the IV.  The doctor tries to assure me it's OK, but I'm not sure.  To appease me, he finally decides to start with oral rehydration salts and a rice mixture to calm Dax's stomach.  In a little over an hour, Dax's color is coming back. Two hours later he's able to stand and walk on his own. 

It's like waking from a bad dream.  As my adrenalin tapers off, the immense relief leaves me feeling weak--like I have just been beaten up.  I apologize to Dr. Rasheed for being so paranoid. He very kindly says, ``I understand''.  (I later learned it would take 10 to 20 times that much air in the IV line to cause any problem.)

Our time in the Kalash Valley now seems a world away.  However, the episode with the shaman Janduli, especially his last words, come flashing back to me.  `You will be safe.'
 
Thinking back to that day I remember arriving at Janduli's camp in the late afternoon after spending all day climbing 4,300 feet up to his mountain pasture.  For the last quarter mile, we followed an increasing odor and eventually overpowering stench of goat urine to his camp.  Wearing a wide, warm smile, Janduli greeted us, then enthusiastically waved us on into his little rock hut for tea.  Inside a small fire was burning.  A good size hole in the wall let in a shaft of light that lit up the smoke as it made its escape.  However, it was the goat urine and hordes of flies that proved unbearable. We apologized and headed outside to conduct our interview on a huge slab of granite overlooking the mountain range in Eastern Afghanistan. 

 Janduli had the gift of gab which kept Yasir, our translator, struggling to keep up.  `He had 60 goats and sheep; snow leopards were his biggest problem; smartest animals on earth; his shamanic calling was difficult; many visions and voices; he thought he was dying; his father (a Dehar) mentored his initiation.'  He continued talking non-stop for well over an hour then stood up and declared he needed to do a ceremony for us in the morning.  I immediately tried to discourage him.  Yasir had told me Kalash shaman typically sacrifice an animal before going into trance.  Killing one of his animals was the last thing I wanted him to do.  But, despite my repeated requests, he kept insisting, ``You have traveled so far. I must bless your journey.   My spirits are demanding it. I will do it in the morning.  I must!'' I tried again but he wasn't about to be dissuaded. 

That night Dax and I slept out in the open on a huge flat boulder illuminated by a heavenly dome of tightly packed stars and the brightest Milky Way I had ever seen.  Here we were in another world silently staring at the countless number of other worlds above us. I was spellbound in an unforgettable moment of awe.  

Early the next morning Janduli's sons started a fire of juniper branches, sacrificed one of the sheep, then poured the thick red blood over the burning branches while Janduli, hands high in the air, sent prayers to the mountain spirits. I thought to myself, This is all for us. We're a greater threat to his flock than the snow leopards. 

He turned his head toward the smoldering branches and started inhaling the smoke until his eyes closed--then collapsed straight back into the arms of his sons.  He remained in a trance, cradled in their arms, for about four or five minutes before they carried his limp body into his hut. That was the last we saw of him. 

Much later as we were getting ready to leave, I had Yasir ask one of the sons once again how Janduli was doing.  Was he alright?  He told us his father was okay and his only words after the ceremony were, ``Your journey will be difficult, but you will be safe.'' That was it, nothing more.  We left without being able to see him or say good-bye.

Now 200 miles away as we express our gratitude to Dr. Rasheed, Janduli's prediction is echoing in my head.  You will be safe….You will be safe.  I'm still in a daze as we get in the Jeep and drive off. Dax has lost some weight but acts and looks normal again. As we rattle along on this noisy, dusty road my mind is trying to make sense of what just occurred. Could Janduli's ceremony and spirit guides have anything to do with what just happened?  Was he able to step out of time in his trance state and predict--or even direct--the future? Or was his somewhat vague prediction just a highly unlikely coincidence?  That certainly is the most rational explanation—and the easiest to understand. 

  After documenting indigenous and tribal cultures worldwide for over two decades I've often wondered:  Why have these cultures used trance-like states for centuries to do healing and visionary work?  If a shamanic trance doesn't have any useful power, why has it been  practiced worldwide for millennia? 

  However, right now I don't need answers.  I still feel shaky. There's a tingling chill in my back and arms. I'm simply filled with gratitude and wonder at what has just happened. 

Now twenty years later as I write I realize that some of my awe inspiring experiences, like that in Pakistan, have begun with a serious dose of fear. It seems if my understanding of the world is challenged enough, and if I'm able to pause and embrace the unknown, I can often lose myself into a transcendent feeling of wonder—a feeling I've grown to appreciate. What started as frightening spiritual encounters when I was a child has evolved over the years into a compelling curiosity. 

 When I examine how events have played out in my life, even though I don't fully understand them, it seems they have been coaxing me to be more open to spiritual experiences and mystical beliefs that I previously would never have entertained.  


\end{spacing}
\end{document}